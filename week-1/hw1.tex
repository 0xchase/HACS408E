% A (minimal) template for problem sets and solutions using the exam document class
\documentclass[answers]{exam}
\usepackage{amsmath}
\usepackage{amsthm}
\usepackage{amsfonts}
\usepackage{amssymb}
\usepackage{mathrsfs}
\renewcommand{\qedsymbol}{$\blacksquare$}
%% \union - Example: \union{j \in J}{A_j}
\newcommand{\union}[2]{\underset{#1}\bigcup #2}
%% \inter - like \union, but with \bigcap
\newcommand{\inter}[2]{\underset{#1}\bigcap #2}
\usepackage{graphicx}
\graphicspath{ {./} }
\usepackage{hyperref}
\hypersetup{
colorlinks=true,
linkcolor=blue,
filecolor=magenta,
urlcolor=cyan,
}
%%%% Useful stuff %%%%
%%% Insert URL
% \href{URL}{DISPLAY TEXT}
%%% Insert image, scale=0.5 => 1/2 image original size
% \includegraphics[scale=0.5]{IMAGE}
%%% Bold text
% \textbf{TEXT}
%%% Code or console output
%\begin{verbatim}
% TEXT HERE
%\end{verbatim}
\begin{document}
%% Title
\begin{center}
\Large HACS408E: Homework 1: Static Analysis \\
Name: %name here
\end{center}
%Part 1
\section*{Survey}
The \texttt{BINARY_NAME} binary is an executable that will NOT harm your
computer if ran. Be as detailed as possible in your answers, the more the merrier!
(Use complete sentences).
%%%%%%%%%%%%%%%%%%%%%
\begin{questions}
\question{
What type of file is this?
}
\begin{solution}
\end{solution}
%%%%%%%%%%%%%%%%%%%%%%%
\question{
Is it a console or a GUI Program?
}
\begin{solution}
\end{solution}
%%%%%%%%%%%%%%%%%%%%%%%%
\question{
Is the checksum valid?
}
\begin{solution}
\end{solution}
%%%%%%%%%%%%%%%%%%%%%%%
\question{
What OS ABI is defined in this executable?
}
\begin{solution}
\end{solution}
%%%%%%%%%%%%%%%%%%%%%%%
\question{
Does the executable have any dynamic libraries loaded, if so what are they?
}
\begin{solution}
\end{solution}
%%%%%%%%%%%%%%%%%%%%%%%
\question{
What section does the entry point fall in? Is this unusual, why, and if so,
when in the real world would this be the case?
}
\begin{solution}
\end{solution}
%%%%%%%%%%%%%%%%%%%%%%%
\question{
What can be determined from the strings?
}
\begin{solution}
\end{solution}
%%%%%%%%%%%%%%%%%%%%%%%
\question{
What function does the main return call come from?
}
\begin{solution}
\end{solution}
%%%%%%%%%%%%%%%%%%%%%%%
\question{
What section header number and name are you able to write to?
}
\begin{solution}
\end{solution}
%%%%%%%%%%%%%%%%%%%%%%%
\question{
Is this file packed? If so, which packer was used?
}
\begin{solution}
\end{solution}
%%%%%%%%%%%%%%%%%%%%%%%
\question{
What compiler do you think was used to build this executable?
}
\begin{solution}
\end{solution}
\end{questions}
\section*{Analysis}
Your objective is to find a way for the binary to execute the function that
prints the "You made it!" message. Use any Reverse Engineering tool(s) of your
choice. Feel free to turn off ASLR, note this in the write up. Write your solution
below in detail and explain how and why your solution achieves the objective.
\begin{solution}
\end{solution}
\end{document}

